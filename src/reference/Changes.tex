
\documentclass[a4paper,11pt,twoside]{article}

\usepackage{scaladoc}
\usepackage{fleqn}
\usepackage{modefs}
\usepackage{math}
\usepackage{scaladefs}

\ifpdf
    \pdfinfo {
        /Author   (Martin Odersky)
        /Title    (The Scala Language Specification)
        /Keywords (Scala)
        /Subject  ()
        /Creator  (TeX)
        /Producer (PDFLaTeX)
    }
\fi


\renewcommand{\todo}[1]{}
\newcommand{\notyet}[1]{\footnote{#1 not yet implemented.}}
\newcommand{\Ts}{\mbox{\sl Ts}}
\newcommand{\tps}{\mbox{\sl tps}}
\newcommand{\psig}{\mbox{\sl psig}}
\newcommand{\fsig}{\mbox{\sl fsig}}
\newcommand{\csig}{\mbox{\sl csig}}
\newcommand{\args}{\mbox{\sl args}}
\newcommand{\targs}{\mbox{\sl targs}}
\newcommand{\enums}{\mbox{\sl enums}}
\newcommand{\proto}{\mbox{\sl pt}}
\newcommand{\argtypes}{\mbox{\sl Ts}}
\newcommand{\stats}{\mbox{\sl stats}}
\newcommand{\overload}{\la\mbox{\sf and}\ra}
\newcommand{\op}{\mbox{\sl op}}

\newcommand{\ifqualified}[1]{}
\newcommand{\iflet}[1]{}
\newcommand{\ifundefvar}[1]{}
\newcommand{\iffinaltype}[1]{}
\newcommand{\ifpackaging}[1]{}
\newcommand{\ifnewfor}[1]{}

\newcommand{\U}[1]{\mbox{$\backslash$u{#1}}}
\newcommand{\Urange}[2]{\mbox{$\backslash$u{#1}-$\backslash$u{#2}}}
%\newcommand{\U}[1]{\mbox{U+{#1}}}

\title{Changes between Scala Version 1.0 and 2.0}

\author{Martin Odersky, EPFL}

\begin{document}
\sloppy
\maketitle

Scala in its second version is different in some details from the
first version of the language. There have been several additions and
some old idioms are no longer supported. However, it is usually fairly
simple to modify old Scala programs so that they compile under version
2.0. Furthermore, the necessary modifications can usually done in a
backwards compatible way, so that the same Scala source program is
accepted by the old as well as new version of the compiler.

The Scala reference compiler supports a new command line option,
\lstinline@-migrate@, which assists in migrating Scala 1.0
sources to version 2.0.

In the following, the main changes between Scala 1.0 and Scala 2.0 are
described. These changes can be summed up as follows:
\begin{itemize}
\item
There are three new keywords (Section~\ref{keys}).
\item
Newlines can now be used as statement separators, making semicolons
optional (Section~\ref{newlines}).
\item 
Some old syntax constructs are no longer legal (Section~\ref{restrictions}).
\item
The recommended syntax of selftype annotations has changed
(Section~4).
\item 
For comprehensions now admit embedded definitions (Section~\ref{fors})
\item
The implicit conversion rules from methods to values have changed
(Section~\ref{conv}).
\item
Getters and setters can now be generated for class parameters
(Section~\ref{classparams}).
\item
Visibility of classes and their members can now be restricted in more
finegrained ways (Section~\ref{privatequals}).
\item
The model of mixin composition has changed (Section~\ref{traits}).
\item
Views have been generalized to the new concept of implicit parameters
(Section~\ref{implicits}).
\item
Typing rules for pattern matching have been relaxed. (Section~\ref{gadts}).
\end{itemize}


\section{New Keywords}\label{keys}

The following three words are now reserved; they cannot be used as
identifiers.
\begin{lstlisting}
implicit    match     requires
\end{lstlisting}

\section{Newlines as Statement Separators}\label{newlines}

Newlines can now be used as statement separators in place of
semicolons. A newline is significant as a statement separator if the
following three conditions are fulfilled:
\begin{enumerate}
\item
The token immediately preceding the newline can terminate a statement.
\item
The token immediately following the newline can begin a statement.
\item
The token appears in a region where multiple statements are allowed.
\end{enumerate}
Multiple statements are allowed for the program as a whole and between
\lstinline@{ ... }@ braces; they are disallowed between \lstinline@(...)@
parentheses and between \lstinline@[...]@ brackets.

For instance, a semicolon is now no longer necessary after the
\lstinline@println@ call in the function below: 
\begin{lstlisting}
def square(x: int) = {
  println("square called")
  x * x
}
\end{lstlisting}
This new convention makes some previously allowed multi-line expressions illegal.  
An example is:
\begin{lstlisting}
def inAllowedRange(x: int) = 
  x >= loBound &&       // newline significant here!
  x <= hiBound
\end{lstlisting}
In these cases it suffices to enclose the multi-line expression in
parentheses to suppress the generation of a newline token:
\begin{lstlisting}
def inAllowedRange(x: int) = (
  x >= loBound &&       // no newline is inserted
  x <= hiBound
)
\end{lstlisting}


\section{Syntax Restrictions}\label{restrictions}

There are some other situations where old constructs no longer work:

\subsection*{Pattern matching expressions} 

The \lstinline@match@
keyword now appears only as infix operator between a selector
expression and a number of cases, as in:
\begin{lstlisting}
  expr match {
    case Some(x) => ...
    case None => ...
  }
\end{lstlisting}
Variants such as \lstinline@ expr.match {...} @ 
or just
\lstinline@ match {...} @
are no longer supported.

\subsection*{``With'' in extends clauses}

The idiom
\begin{lstlisting}
class C with M { ... }
\end{lstlisting}
is no longer supported. A \lstinline@with@ connective is only allowed
following an \lstinline@extends@ clause. For instance, the line
above would have to be written
\begin{lstlisting}
class C extends AnyRef with M { ... } .
\end{lstlisting}
However, assuming \lstinline@M@ is a trait (see
\ref{traits}), it is also legal to write
\begin{lstlisting}
class C extends M { ... }
\end{lstlisting}
The latter expression is treated as equivalent to
\begin{lstlisting}
class C extends S with M { ... }
\end{lstlisting}
where \lstinline@S@ is the superclass of \lstinline@M@.

\subsection*{Regular Expression Patterns} 

The only form of regular
expression pattern that is currently supported is a sequence pattern,
which might end in a sequence wildcard \code{_*}. Example:
\begin{lstlisting}
case List(1, 2, _*) => ... // will match all lists starting with \code{1,2}.
\end{lstlisting}
It is at current not clear whether this is a permanent restriction. We
are evaluating the possibility of re-introducing full regular
expression patterns in Scala.

\section{Selftype Annotations}\label{self}

The recommended syntax of selftype annotations has changed. 
\begin{lstlisting}
class C: T extends B { ... }
\end{lstlisting}
becomes
\begin{lstlisting}
class C requires T extends B { ... }
\end{lstlisting}
That is, selftypes are now indicated by the new \lstinline@requires@
keyword. The previous syntax is still available but is considered deprecated. 

\section{For-comprehensions}\label{fors}

For-comprehensions now admit embedded value and pattern definitions. Example:
\begin{lstlisting}
for {
  val x <- List.range(1, 100)
  val y <- List.range(1, x)
  val z = x + y
  isPrime(z)
} yield Pair(x, y)
\end{lstlisting}
Note the definition ~\lstinline@val z = x + y@ as the third item in
the for-comprehension. 

\section{Conversions}\label{conv}

The rules for implicit conversions of methods to functions have been
tightened. Previously, a parameterized method used as a value was always
implicitly converted to a function. This could lead to unexpected
results when method arguments where forgotten. Consider for instance
the statement below:
\begin{lstlisting}
show(x.toString)
\end{lstlisting}
where \lstinline@show@ is defined as follows:
\begin{lstlisting}
def show(x: String) = Console.println(x) .
\end{lstlisting}
Most likely, the programmer forgot to supply an empty argument list
\lstinline@()@ to \lstinline@toString@. The previous Scala version would
treat this code as a partially applied method, and expand it to:
\begin{lstlisting}
show(() => x.toString())
\end{lstlisting}
As a result, the address of a closure would be printed instead of the
value of \lstinline@s@.

Scala version 2.0 will apply a conversion from partially applied
method to function value only if the expected type of the expression
is indeed a function type. For instance, the conversion would not be
applied in the code above because the expected type of
\lstinline@show@'s parameter is \lstinline@String@, not a function
type. 

The new convention disallows some previously legal code. Example:
\begin{lstlisting}
def sum(f: int => double)(a: int, b: int): double =
  if (a > b) 0 else f(a) + sum(f)(a + 1, b)

val sumInts  =  sum(x => x)  // error: missing arguments
\end{lstlisting}
The partial application of \lstinline@sum@ in the last line of
the code above will not be converted to a function type. Instead, the
compiler will produce an error message which states that arguments for method
\lstinline@sum@ are missing. The problem can be fixed by providing an
expected type for the partial application, for instance by annotating
the definition of \lstinline@sumInts@ with its type:
\begin{lstlisting}
val sumInts: (int, int) => double  =  sum(x => x)  // OK
\end{lstlisting}

On the other hand, Scala 2 now automatically applies methods
with empty parameter lists to \lstinline@()@ argument lists when
necessary. For instance, the \lstinline@show@ expression above will
now be expanded to
\begin{lstlisting}
show(x.toString()) .
\end{lstlisting}
This is probably what the user expected all along.

The new Scala version also relaxes the rules of overriding with
respect to empty parameter lists. It is now possible to override a
method with an explicit, but empty parameter list \lstinline@()@ with
a parameterless method, and {\em vice versa}. For instance, 
the following class definition is now legal:
\begin{lstlisting}
class C {
  override def toString: String = ...
}
\end{lstlisting}
Previously this definition would have been rejected, because the
\lstinline@toString@ method as inherited from
\lstinline@java.lang.Object@ takes an empty parameter list.  

\section{Class Parameters}\label{classparams}

A class parameter may now be prefixed by \lstinline@val@ or
\lstinline@var@. Example:
\begin{lstlisting}
class C(x: int, val y: String, var z: List[String])
val c = new C(1, "abc", List())
c.z = c.y :: c.z
\end{lstlisting}
If a class parameter is prefixed with \lstinline@val@, it is
accessible via a parameterless getter method in instances of the
class.  In the example above, it is this legal to access the second
parameter in \lstinline@c.y@, whereas \lstinline@c.x@ would be
illegal, because the \lstinline@x@ parameter is not declared using
\lstinline@val@. The treatment of \lstinline@val@ parameters is thus
analogous to the treatment of parameters in a case class. In other words,
case class parameters are always implicitly prefixed with \lstinline@val@.

If a class parameter is prefixed with \lstinline@var@, the class in
question obtains additionally a setter method to change the value of
the corresponding field. Note for instance the \lstinline@z@ parameter
in the example above which is read as well as written.

Parameters that are introduced with \lstinline@val@ or \lstinline@var@
may also carry modifiers. Example:
\begin{lstlisting}
class C(protected var x: String)
\end{lstlisting}
Here, accesses to \lstinline@x@ (and its setter, \lstinline@x_=@), are
restricted to subclasses of \lstinline@C@.

\section{Private Qualifiers}\label{privatequals}

Previously, Scala had three levels of visibility: {\em private},
{\em protected} and {\em public}. There was no way to
restrict accesses to members of the current package, as in Java. Scala
2 now defines access qualifiers that let one express this level of
visibility, among others. In the definition
\begin{lstlisting}
private[C] def f(...)
\end{lstlisting}
access to \lstinline@f@ is restricted to all code within the class or
package \lstinline@C@ (which must contain the definition of \lstinline@f@).

Consider the following example:
\begin{lstlisting}
package outerpkg.innerpkg
class Outer {
  abstract class Inner {
    private[Outer] def f()
    private[innerpkg] def g()
    private[outerpkg] def h()
  }
}
\end{lstlisting}
Here, accesses to the method \lstinline@f@ can appear anywhere within
\lstinline@OuterClass@, but not outside it. Accesses to method
\lstinline@g@ can appear anywhere within the package
\lstinline@outerpkg.innerpkg@, as would be the case for
package-private methods in Java. Finally, accesses to method
\lstinline@h@ can appear anywhere within package \lstinline@outerpkg@,
including packages contained in it. 

\section{Changes in the Mixin Model}\label{traits}

The model which details mixin composition of classes has changed
significantly. The main differences are:
\begin{enumerate}
\item
We now distinguish between {\em traits} that are used as mixin classes
and normal classes. The syntax of traits has been generalized from
version 1.0, in that traits are now allowed to have mutable
fields. However, as in version 1.0, traits may still do not have
constructor parameters.
\item
Member resolution and super accesses are now both defined in terms of
a {\em class linearization}. 
\item
Scala's notion of method overloading has been generalized; in
 particular, it is now possible to have overloaded variants of the
 same method in a subclass and in a superclass, or in several different
 mixins. This makes method overloading in Scala conceptually the
 same as in Java.
\end{enumerate}

The new mixin model is explained in more detail in the rest of this
section, which is adapted from the paper \cite{odersky05sca}.

Mixin class composition in Scala is a fusion of the object-oriented,
linear mixin composition from Bracha~\cite{bracha90mixinbased}, and
the more symmetric approaches of mixin
\cite{duggan96mixins,hirschowitz-leory:02mixin} and traits
\cite{schaerli03traits}.  To start with an example, consider the
following abstraction for iterators.
\begin{lstlisting}
abstract class AbsIterator {
  type T
  def hasNext: boolean
  def next: T
}
\end{lstlisting}
Next, consider a mixin class which extends \lstinline@AbsIterator@
with a method \lstinline@foreach@, which applies a given function to
every element returned by the iterator.
\begin{lstlisting}
trait RichIterator extends AbsIterator {
  def foreach(f: T => unit): unit = 
    while (hasNext) f(next)
}
\end{lstlisting}
Here is a concrete iterator class, which returns successive characters
of a given string:
\begin{lstlisting}
class StringIterator(s: String) extends AbsIterator {
  type T = char
  private var i = 0
  def hasNext = i < s.length()
  def next = { val x = s.charAt(i); i = i + 1; x }
}
\end{lstlisting}
We now would like to combine the functionality of
\lstinline@RichIterator@ and \lstinline@StringIterator@ in a single
class. With single inheritance and interfaces alone this is
impossible, as both classes contain member implementations with code.
Therefore,
Scala provides a \emph{mixin-class composition} mechanism which
allows programmers to reuse the delta of a class definition, i.e., all
new definitions that are not inherited. This mechanism makes it
possible to combine \lstinline@RichIterator@ with
\lstinline@StringIterator@, as is done in the following test program.
The program prints a column of all the characters of a given string.
\begin{lstlisting}
object Test {
  def main(args: Array[String]): unit = {
    class Iter extends StringIterator(args(0)) with RichIterator
    val iter = new Iter
    iter foreach System.out.println 
  }
}
\end{lstlisting}
The \lstinline@Iter@ class in function \lstinline@main@ is constructed
from a mixin composition of the parents \lstinline@StringIterator@ and
\lstinline@RichIterator@. The first parent is called the {\em
superclass} of \lstinline@Iter@, whereas the second parent is called a
{\em mixin}. 
\medskip

\subsection*{Class Linearization}

The classes reachable through transitive closure of the 
inheritance relation from a class $C$ are called the {\em base
classes} of $C$.  Because of mixins, the inheritance relationship on
base classes forms in general a directed acyclic graph. A
linearization of this graph is defined as follows.

\newcommand{\uright}{\;\vec +\;}
\newcommand{\lin}[1]{{\cal L}(#1)}

\begin{definition}\label{def:lin} Let $C$ be a class with parents 
\lstinline@$C_n$ with ... with $C_1$@. 
The {\em class linearization} of $C$, $\lin C$ is defined as follows:
\bda{rcl}
\lin C &=& \{C\} \uright \lin{C_1} \uright \ldots \uright \lin{C_n} 
\eda
Here $\uright$ denotes concatenation where elements of the right operand
replace identical elements of the left operand:
\bda{lcll}
\{a, A\} \uright B &=& a, (A \uright B)  &{\bf if} a \not\in B \\
                 &=& A \uright B       &{\bf if} a \in B
\eda
\end{definition}
Conceptually, it is easiest to construct the base class linearization
of a class $C$ from back to front. We start with the linearization of
$C$'s superclass, which forms the tail of the linearization of $C$.
We then prepend to this all classes of the linearization of $C$'s
first mixin class (if there is a mixin class), except for those
classes that are already included by the linearization of the
superclass. We then continue with this scheme for all other mixin
classes of $C$, adding at each step only classes that are not already
in the constructed linearization. Finally, the first class in the
linearization of $C$ is always $C$ itself.

For instance, the linearization of class \lstinline@Iter@ is
\begin{lstlisting}
{ Iter, RichIterator, StringIterator, AbsIterator, AnyRef, Any } .
\end{lstlisting}
The linearization of a class refines the inheritance relation: if
$C$ is a subclass of $D$, then $C$ precedes $D$ in any linearization
where both $C$ and $D$ occur.
\ref{def:lin} also satisfies the property that a linearization
of a class always contains the linearization of its direct superclass
as a suffix.  For instance, the linearization of
\lstinline@StringIterator@ is
\begin{lstlisting}
{ StringIterator, AbsIterator, AnyRef, Any } .
\end{lstlisting}
This is a suffix of the linearization of its subclass \lstinline@Iter@.
The same is not true for the linearization of mixin classes.
\comment{
For instance, the linearization of \lstinline@RichIterator@ is
\begin{lstlisting}
{ RichIterator, AbsIterator, AnyRef, Any }
\end{lstlisting}
which is not a suffix of the linearization of \lstinline@Iter@.
}
It is also possible that classes of the linearization of a trait
appear in different order in the linearization of an inheriting class, i.e.\ 
linearization in Scala is not monotonic \cite{c3-dylan}.

\subsection*{Membership}

The \lstinline@Iter@ class inherits members from both
\lstinline@StringIterator@ and \lstinline@RichIterator@. Generally, a
class derived from a mixin composition 
\lstinline@$C_n$ with ... with $C_1$@ can define members itself and can
inherit members from all parent classes. Scala adopts Java and C\#'s
conventions for static overloading of methods. It is thus possible
that a class defines and/or inherits several methods with the same
name\footnote{One might disagree with this design choice because of its
complexity, but it is necessary to ensure interoperability, for instance
when inheriting from Java's Swing libraries.}.
To decide whether a defined member of a class $C$ overrides a
member of a parent class, or whether the two co-exist as overloaded
variants in $C$, Scala uses the following definition of {\em matching}
on members, which is derived from similar concepts in Java and C\#:

\begin{definition}
A member definition $M$ {\em matches} a member definition $M'$, if $M$
and $M'$ bind the same name, and one of following holds.
\begin{enumerate}
\item Neither $M$ nor $M'$ is a method definition.
\item $M$ and $M'$ define both monomorphic methods with equal argument types.
\item $M$ and $M'$ define both polymorphic methods with 
equal number of argument types $\overline T$, $\overline T'$
and equal numbers of type parameters
$\overline t$, $\overline t'$, say, and $\overline T' = [\overline t'/\overline t]\overline T$.
%every argument type
%$T_i$ of $M$ is equal to the corresponding argument type $T'_i$ of
%$M'$ where every occurrence of a type parameter $t'$ of $M'$ has been replaced by the corresponding type parameter $t$ of $M$.
\end{enumerate}
\end{definition}
Member definitions of a class fall into two categories: concrete and
abstract.  There are two rules that determine the set of members of a
class, one for each category:

\begin{definition}\label{def:member}
A {\em concrete member} of a class $C$ is any concrete definition $M$ in
some class $C_i \in \lin C$, except if there is a preceding class $C_j
\in \lin C$ where $j < i$ which defines a concrete member $M'$ matching $M$.  

An {\em abstract member} of a class $C$ is any abstract definition $M$
in some class $C_i \in \lin C$, except if $C$ contains already a
concrete member $M'$ matching $M$, or if there is a preceding class
$C_j \in \lin C$ where $j < i$ which defines an abstract member $M'$ matching
$M$.
\end{definition}
This definition also determines the overriding relationships between
matching members of a class $C$ and its parents.  First, a concrete
definition always overrides an abstract definition.  Second, for
definitions $M$ and $M$' which are both concrete or both abstract, $M$
overrides $M'$ if $M$ appears in a class that precedes (in the
linearization of $C$) the class in which $M'$ is defined.

\subsection*{Super calls}
Consider the following class of synchronized iterators, which ensures
that its operations are executed in a mutually exclusive way when
called concurrently from several threads.
\begin{lstlisting}
trait SyncIterator extends AbsIterator {
  abstract override def hasNext: boolean = 
    synchronized(super.hasNext)
  abstract override def next: T = 
    synchronized(super.next)
}
\end{lstlisting}
To obtain rich, synchronized iterators over strings, one uses a mixin
composition involving three classes:
\begin{lstlisting}
StringIterator(someString) with RichIterator 
                           with SyncIterator
\end{lstlisting}
This composition inherits the two members \lstinline@hasNext@ and
\lstinline@next@ from the trait \lstinline@SyncIterator@.  Each
method wraps a \lstinline@synchronized@ application around a call to
the corresponding member of its superclass. 

Because \lstinline@RichIterator@ and \lstinline@StringIterator@ define
different sets of members, the order in which they appear in a mixin
composition does not matter.
In the example above, we could have equivalently written
\begin{lstlisting}
StringIterator(someString) with SyncIterator 
                           with RichIterator
\end{lstlisting}
There's a subtlety, however.  The class accessed by the
\code{super} calls in \lstinline@SyncIterator@ is not its
statically declared superclass \lstinline@AbsIterator@. This would not
make sense, as \lstinline@hasNext@ and \lstinline@next@ are abstract
in this class.  Instead, {\em super} accesses the superclass
\lstinline@StringIterator@ of the mixin composition in which
\lstinline@SyncIterator@ takes part. In a sense, the superclass in a
mixin composition {\em overrides} the statically declared superclasses
of its mixins. It follows that calls to {\em super} cannot be
statically resolved when a class is defined; their resolution has to
be deferred to the point where a class is instantiated or inherited.
This is made precise by the following definition.

\begin{definition}
Consider an expression \lstinline@super.$M$@ in a base class $C$ of
$D$.  To be type correct, this expression must refer statically to
some member $M$ of a parent class of $C$. In the context of $D$, the
same expression then refers to a member $M'$ which matches $M$, and
which appears in the first possible class that follows $C$ in the
linearization of $D$.
\end{definition}
Note finally that in a language like Java or C\#, the
{\em super} calls in class \lstinline@SyncIterator@ would be
illegal, precisely because they designate abstract members of the
static superclass. As we have seen, Scala allows this construction,
but it still has to make sure that the class is only used in a context
where {\em super} calls access members that are concretely
defined. This is enforced by the occurrence of the
\lstinline@abstract@ and
\lstinline@override@ modifiers in class \lstinline@SyncIterator@. An
\lstinline@abstract override@ modifier pair in a method definition
indicates that the method's definition is not yet complete because it
overrides and uses an abstract member in a superclass. A class with
incomplete members must be declared abstract itself, and subclasses of
it can be instantiated only once all members overridden by such
incomplete members have been redefined.

Calls to {\em super} may be threaded so that they follow the class
linearization (this is a major difference between Scala 2's mixin
composition and multiple inheritance schemes). For example, consider
another class similar to \lstinline@SyncIterator@ which prints all
returned elements on standard output.
\begin{lstlisting}
trait LoggedIterator extends AbsIterator {
  abstract override def next: T = { 
    val x = super.next; System.out.println(x); x 
  }
}
\end{lstlisting}
One can combine synchronized with logged iterators in a mixin
composition:\medskip

\begin{lstlisting}
class Iter2 extends StringIterator(someString) 
               with SyncIterator with LoggedIterator;         
\end{lstlisting}
The linearization of \lstinline@Iter2@ is
\begin{lstlisting}
{ Iter2, LoggedIterator, SyncIterator, 
  StringIterator, AbsIterator, AnyRef, Any }
\end{lstlisting}
Therefore, class \lstinline@Iter2@ inherits 
its \lstinline@next@ method from class \lstinline@LoggedIterator@,
the \lstinline@super.next@ call in this method refers
to the \lstinline@next@ method in class \lstinline@SyncIterator@, whose  
\lstinline@super.next@ call finally refers to the
\lstinline@next@ method in class \lstinline@StringIterator@.

If logging should be included in the synchronization, this can be
achieved by reversing the order of the mixins:
\begin{lstlisting}
class Iter2 extends StringIterator(someString) 
               with LoggedIterator with SyncIterator;         
\end{lstlisting}
In either case, calls to \lstinline@next@ follow via
{\em super} the linearization of class \lstinline@Iter2@.

\section{Implicit Parameters}\label{implicits}

Views in Scala 1.0 have been replaced by the more general concept of
implicit parameters. Implicit parameters let one write generic code
analogous to Haskell's type classes \cite{chen-hudak-odersky:parametric-conf},
or, in a C++ context, to Siek and Lumsdaine's ``concepts'' \cite{siek-lumsdaine:pldi05}.

\subsection*{Motivation}

As an example, let's start with an abstract class of semi-groups that
support an unspecified \lstinline@add@ operation.
\begin{lstlisting}
abstract class SemiGroup[a] {
  def add(x: a, y: a): a
}
\end{lstlisting}
Here's a subclass \lstinline@Monoid@ of \lstinline@SemiGroup@ which adds a
\lstinline@unit@ element.
\begin{lstlisting}
abstract class Monoid[a] extends SemiGroup[a] {
  def unit: a
}
\end{lstlisting}
Here are two implementations of monoids:
\begin{lstlisting}
object stringMonoid extends Monoid[String] {
  def add(x: String, y: String): String = x.concat(y)
  def unit: String = ""
}

object intMonoid extends Monoid[int] {
  def add(x: Int, y: Int): Int = x + y
  def unit: Int = 0
}
\end{lstlisting}
A \lstinline@sum@ method, which works over arbitrary
monoids, can be written in plain Scala as follows.
\begin{lstlisting}
def sum[a](xs: List[a])(m: Monoid[a]): a =
  if (xs.isEmpty) m.unit
  else m.add(xs.head, sum(xs.tail)(m))
\end{lstlisting}
This \lstinline@sum@ method can be called as follows:
\begin{lstlisting}
sum(List("a", "bc", "def"))(stringMonoid)
sum(List(1, 2, 3))(intMonoid)
\end{lstlisting}
All this works, but it is not very nice. The problem is that the
monoid implementations have to be passed into all code that uses them.
We would sometimes wish that the system could figure out the correct
arguments automatically, similar to what is done when type arguments
are inferred. This is what implicit parameters provide.

\subsection*{Implicit Parameters: The Basics}

In Scala 2 there is a new \lstinline@implicit@ keyword that can be
used at the beginning of a parameter list. Syntax:
\begin{lstlisting}
ParamClauses ::= {`(' [Param {`,' Param}] ')'} 
                 [`(' implicit Param {`,' Param} `)']
\end{lstlisting}
If the keyword is present, it makes all parameters in the list
implicit.  
For instance, the following version of \lstinline@sum@ has
\lstinline@m@ as an implicit parameter.
\begin{lstlisting}
def sum[a](xs: List[a])(implicit m: Monoid[a]): a = 
  if (xs.isEmpty) m.unit
  else m.add(xs.head, sum(xs.tail))
\end{lstlisting}
As can be seen from the example, it is possible to combine normal and
implicit parameters. However, there may only be one implicit parameter
list for a method or constructor, and it must come last.

\lstinline@implicit@ can also be used as a modifier for definitions
and declarations. Examples:

\begin{lstlisting}
implicit object stringMonoid extends Monoid[String] {
  def add(x: String, y: String): String = x.concat(y)
  def unit: String = ""
}
implicit object intMonoid extends Monoid[int] {
  def add(x: Int, y: Int): Int = x + y
  def unit: Int = 0
}
\end{lstlisting}

The principal idea behind implicit parameters is that arguments for
them can be left out from a method call. If the arguments
corresponding to an implicit parameter section are missing, they are
inferred by the Scala compiler.

The actual arguments that are eligible to be passed to an implicit
parameter are all identifiers X that can be accessed at the point
of the method call without a prefix and that denote an implicit
definition or parameter.

If there are several eligible arguments which match the implicit
parameter's type, the Scala compiler will chose a most specific one,
using the standard rules of static overloading resolution.
For instance, assume the call
\begin{lstlisting}
  sum(List(1, 2, 3))
\end{lstlisting}
in a context where \lstinline@stringMonoid@ and \lstinline@intMonoid@
are visible.  We know that the formal type parameter \lstinline@a@ of
\lstinline@sum@ needs to be instantiated to \lstinline@Int@. The only
eligible value which matches the implicit formal parameter type
\lstinline@Monoid[Int]@ is \lstinline@intMonoid@ so this object will
be passed as implicit parameter.

This discussion also shows that implicit parameters are inferred after
any type arguments are inferred. 

\subsection*{Implicit Conversions}

Say you have an expression $E$ of type $T$ which is expected to type
$S$. $T$ does not conform to $S$ and is not convertible to $S$ by
some other predefined conversion. Then the Scala compiler will try to
apply as last resort an implicit conversion $I(E)$. Here, $I$ is an
identifier denoting an implicit definition or parameter that is
accessible without a prefix at the point of the conversion, that can
be applied to arguments of type $T$ and whose result type conforms to the
expected type $S$.

Implicit conversions can also be applied in member
selections. Given a selection $E.x$ where $x$ is not a member of the
type $E$, the Scala compiler will try to insert an implicit conversion
$I(E).x$, so that $x$ is a member of $I(E)$.

Implicit conversions replace the views of the previous version of Scala. They have
fundamentally the same functionality. But unlike views, implicit
conversions have user-defined names. This allows a simplification of
the rules for scoping and overloading of implicit conversions.

Here is an example of an implicit conversion function that converts
integers into instances of class \lstinline@scala.Ordered@ (a similar
conversion is defined in module \code{Predef}).
\begin{lstlisting}
implicit def int2ordered(x: int): Ordered[int] = new Ordered[int] {
  def compare(y: int): int = 
    if (x < y) -1 else if (x > y) 1 else 0
}
\end{lstlisting}
If one wants to maintain backwards compatibility with version 1.0 of Scala, one can
also add a view that forwards to \lstinline@int2ordered@:
\begin{lstlisting}
def view(x: int): Ordered[int] = int2ordered(x)
\end{lstlisting}
This view definition will be significant only for version 1.0; for version 2.0
it is just a plain method definition.

\subsection*{View Bounds}

View bounds are still available in Scala 2, but they are now treated as
syntactic sugar for implicit parameters. Consider for instance 
a generic sort method:
\begin{lstlisting}
def sort[a <$\mbox{\%}$ Ordered[a]](xs: List[a]): List[a] = 
  if (xs.isEmpty || xs.tail.isEmpty) xs
  else {
    val Pair(ys, zs) = xs.splitAt(xs.length / 2)
    merge(ys, zs)
  }
\end{lstlisting}
The view bounded type parameter \lstinline@[a <$\mbox{\%}$ Ordered[a]]@
expresses that \lstinline@sort@ is applicable to lists of type
\lstinline@a@ such that there exists an implicit conversion from
\lstinline@a@ to \lstinline@Ordered[a]@. The definition is treated as
a shorthand for the following method signature with an implicit
parameter:
\begin{lstlisting}
def sort[a](xs: List[a])(implicit $c$: a => Ordered[a]): List[a] = ...
\end{lstlisting}
(Here, the parameter name $c$ is chosen arbitrarily in a way
that does not collide with other names in the program.)

As a more detailed example, consider the \lstinline@merge@ method that
comes with the \lstinline@sort@ method above:
\begin{lstlisting}
def merge[a <$\mbox{\%}$ Ordered[a]](xs: List[a], ys: List[a]): List[a] =
  if (xs.isEmpty) ys
  else if (ys.isEmpty) xs
  else if (xs.head < ys.head) xs.head :: merge(xs.tail, ys)
  else ys.head :: merge(xs, ys.tail)
\end{lstlisting}
After expanding view bounds and inserting implicit conversions, this
method implementation becomes:
\begin{lstlisting}
def merge[a](xs: List[a], ys: List[a])
            (implicit $c$: a => Ordered[a]): List[a] =
  if (xs.isEmpty) ys
  else if (ys.isEmpty) xs
  else if (c(xs.head) < ys.head) xs.head :: merge(xs.tail, ys)
  else ys.head :: merge(xs, ys.tail)(c)
\end{lstlisting}
The last two lines of this method definition illustrate two different
uses of the implicit parameter $c$. It is applied in a conversion in
the condition of the second to last line, and it is passed as implicit
argument in the recursive call to \lstinline@merge@ on the last line.

\section{Flexible Typing of Pattern Matching}\label{gadts}

The new version of Scala implements more flexible typing rules when it
comes to pattern matching over heterogeneous class hierarchies. A {\em
heterogeneous class hierarchy} is one where subclasses inherit a
common superclass with different parameter types. An example is the
following class hierarchy of typed terms. 
\begin{lstlisting}
abstract class Term[T]
case class Lit(x: int) extends Term[int]
case class Succ(t: Term[int]) extends Term[int]
case class IsZero(t: Term[int]) extends Term[boolean]
case class If[T](c: Term[boolean], t1: Term[T], t2: Term[T]) extends Term[T]
\end{lstlisting}
Here, there is a common superclass \lstinline@Term[T]@ representing
terms of type \lstinline@T@. This class is refined by alternatives of
numeric literals \lstinline@Lit@, successor function applications
\lstinline@Succ@, an equality with zero test \lstinline@IsZero@, and a
conditional \lstinline@If@, to make a small language of arithmetic and
boolean terms.
The class definitions express also that \lstinline@Lit@ and
\lstinline@Succ@ yield terms of type \lstinline@int@, that
\lstinline@IsZero@ yields a term of type \lstinline@boolean@, and that
\lstinline@If@ yields a term of the same type as its second and third
arguments.

Now, let's write a method to evaluate a term given by the class hierarchy above.
A natural way to do this would be:
\begin{lstlisting}
def eval[a](t: Term[a]): a = t match {
  case Lit(n)        => n
  case Succ(u)       => eval(u) + 1
  case IsZero(u)     => eval(u) == 0
  case If(c, u1, u2) => if (eval(c)) eval(u1) else eval(u2)
}
\end{lstlisting}
In previous Scala, this code would not compile. There were two problems.

The first problem was that every pattern in a pattern match had to have a
type that conforms to the type of the selector expression. In the
\lstinline@eval@ function above, this is not the case. For instance,
the type of the first pattern is \lstinline@Lit@, which does not
conform to the type of the selector expression, \lstinline@Term[a]@.

The first problem could be resolved by widening the type of the selector
expression, i.e.
\begin{lstlisting}
(t: AnyRef) match { ... }
\end{lstlisting}
But another problem still remains: The types of the alternatives of
\lstinline@eval@ do not conform to the method's defined result type.
For instance, the first alternative expression, \lstinline@n@, has
type \lstinline@int@, which does not conform to \lstinline@eval@'s
declared result type \lstinline@a@. The problem could be resolved by
inserting an explicit type cast, as in:
\begin{lstlisting}
(t: AnyRef) match { 
  case Lit(n) => n.asInstanceOf[a]
  ...
}
\end{lstlisting}
However, all these type annotations and casts tend to obscure the
fundamental idea of \lstinline@eval@'s implementation. They are also
provably redundant. To see this, consider again the first pattern,
\lstinline@Lit(n)@. We know that if this pattern matches the selector
\lstinline@t@, then \lstinline@t@'s run-time type must necessarily
conform to \lstinline@Term[int]@. After all, \lstinline@Lit@ is a
declared to be a subtype of this type. Hence, we can deduce that a
successful match of \lstinline@Lit(n)@ will imply the constraint that
\lstinline@a = int@, where \lstinline@a@ is the type parameter of
\lstinline@eval@. Using the same constraint, we can deduce that the
result \lstinline@n@ of the pattern match is indeed a legal member of
the declared result type of \lstinline@eval@, which is again
\lstinline@a@.

Scala 2 permits patterns that are not subtypes of the selector
expression. Furthermore the type checker will deduce for every pattern
a constraint for the type parameters of the selector
expression (if there are any). Compatibility of the case's result with the expected type
of the match-expression is then checked modulo this constraint.  With
these relaxation of the typing rules, method \lstinline@eval@
presented above compiles without error.

These changes make pattern matching in Scala equivalent to pattern
matching over {\em guarded algebraic data types} (GADTs), which
recently have been a popular concept of study in the context of
functional languages \cite{GRDC03}


\bibliographystyle{alpha}
\bibliography{main}

\end{document}

